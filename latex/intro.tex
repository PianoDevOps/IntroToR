\documentclass[10pt]{article}
\usepackage[utf8]{inputenc}
\usepackage[english]{babel} % Language to be used
\usepackage[T1]{fontenc}
\usepackage{amsmath}
\usepackage{amssymb}
\usepackage{amsthm}
\usepackage{cite}  % Package to cite multiple references at once
\usepackage{url}
\usepackage{hyperref}
\usepackage[usenames,dvipsnames]{xcolor}
\usepackage{listings}
\usepackage{graphicx}
\usepackage{enumitem} 
%\usepackage{mathpazo}  Palatino-fonts
\usepackage{txfonts}
\usepackage[stable]{footmisc}

\urlstyle{same} 


\lstset{
  basicstyle=\ttfamily,
  language=python,
  tabsize=4,
  frame=single,
  showstringspaces=false,
  showspaces=false,
  showtabs=false,
  captionpos=b,
  breaklines=true,
  breakatwhitespace=false,        % sets if automatic breaks should only happen at whitespace
  keywordstyle=\color{RoyalBlue},      % keyword style
  commentstyle=\color{ForestGreen},   % comment style
  stringstyle=\color{BrickRed}
}

\hypersetup{%
  colorlinks=true,    % hyperlinks will be black
  linkcolor=green,% hyperlink text will be green
  linkbordercolor=red,% hyperlink borders will be red
  pdfborderstyle={/S/U/W 1}% border style will be underline of width 1pt
}

% Parameters to the margins
\setlength{\topmargin}{-0.50in}
\setlength{\oddsidemargin}{-0.25in}
\setlength{\evensidemargin}{-0.25in}
\setlength{\textwidth}{7.0in}
\setlength{\textheight}{9.00in}

% Here is where all the new theorem environments are defined.
% Their use is described in the text.
\newtheorem{prop}{Proposition}
\newtheorem{thm}{Theorem}[section]
\newtheorem{lem}[thm]{Lemma}
\newtheorem*{Zorn}{Zorn's Lemma}

% Use a new style for definitions
\theoremstyle{definition}
\newtheorem{dfn}{Definition}

% Use a new style for remarks
\theoremstyle{remark}
\newtheorem*{rmk}{Remark}

\title{\centering{Hands-on Intro to R}}

\author{Wim R.\,M.\, Cardoen \& Brett Milash\\
        Center for High-Performance Computing\\
	University of Utah}


\begin{document}
\date{\today}
\maketitle
\thispagestyle{empty}
% ----------------------------------------------------------------
\pagestyle{plain}
\pagenumbering{arabic}
\setcounter{page}{1}
\renewcommand \thesection{\Roman{section}}

\section*{Overview}

This document contains general information for the "Hands-on Introduction to R" lectures.\newline

\section{Access to the \texttt{R} language interpretor and the \texttt{RStudio} IDE}

The lectures require the access to an \texttt{R} language interpretor and the \texttt{RStudio} IDE.
Below, you will find a few options:

\begin{enumerate}
\item If you have a valid CHPC account you can use CHPC's \href{http://ondemand.chpc.utah.edu/}{Ondemand Web Portal}.\newline
      After logging into OnDemand you can launch the \texttt{RStudioServer} application which contains \texttt{R} and \texttt{RStudio}.

\item You can apply for an account on \texttt{RStudioCloud} (which provides $25$ hours of free computing time per month).\newline
      \texttt{RStudioCloud} contains \texttt{R} and \texttt{RStudio}. 

\item You can also install \texttt{R} and \texttt{RStudio} on your machine.\newline
      The \texttt{R} command line interpretor can be downloaded from:
      \begin{itemize}
	      \item either \href{https://mran.microsoft.com/}{Microsoft R Open} (binaries for MS, Linux and Mac-OS).
  	 \item or \href{https://cran.r-project.org/}{The Comprehensive R Archive Network (CRAN)}.
      \end{itemize}			

      The \texttt{RStudio} Desktop IDE can be downloaded \href{https://www.rstudio.com/products/rstudio/download/}{here}.

\end{enumerate}

% Free R cheatsheets:
% https://www.rstudio.com/resources/cheatsheets/


\section{Obtaining the material}
All the course material can be obtained from: \href{https://github.com/wcardoen/IntroToR}{https://github.com/wcardoen/IntroToR}.

If you have \texttt{git} installed, you can obtain the material as follows:
\begin{verbatim}
git clone https://github.com/wcardoen/IntroToR.git
\end{verbatim}
Another option is to download the following \texttt{zip} file: \newline 
\href{https://github.com/wcardoen/IntroToR/archive/refs/heads/main.zip}{https://github.com/wcardoen/IntroToR/archive/refs/heads/main.zip}



\end{document}
