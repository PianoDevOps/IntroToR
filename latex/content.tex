\section*{Overview}

This document contains general information for the "Hands-on Introduction to R" lectures.\newline

\section{Access to the \texttt{R} language interpretor and the \texttt{RStudio} IDE}

The lectures require the access to the \texttt{R} language interpretor as well as the \texttt{RStudio} IDE.

\begin{enumerate}
\item If you have a valid CHPC account you can also use CHPC's \href{http://ondemand.chpc.utah.edu/}{Ondemand Web Portal}.\newline
      After logging into OnDemand you can launch the \texttt{RStudioServer} application which contains \texttt{R} and \texttt{RStudio}.

\item You can apply for an account on \texttt{RStudioCloud} (which provides $25\,h$/month of free computing time).\newline
      \texttt{RStudioCloud} contains \texttt{R} and \texttt{RStudio}. 

\item You can install \texttt{R} and \texttt{RStudio} on your machine.\newline
      The \texttt{R} command line interpretor can be downloaded from:
      \begin{itemize}
	      \item either \href{https://mran.microsoft.com/}{Microsoft R Open} (binaries for MS, Linux and Mac-OS).
  	 \item or \href{https://cran.r-project.org/}{The Comprehensive R Archive Network (CRAN)}.
      \end{itemize}			

      The \texttt{RStudio} Desktop IDE can be downloaded \href{https://www.rstudio.com/products/rstudio/download/}{here}.

\end{enumerate}

% Free R cheatsheets:
% https://www.rstudio.com/resources/cheatsheets/


\section{Obtaining the material}
All the course material can be obtained from: \href{https://github.com/wcardoen/IntroToR}{https://github.com/wcardoen/IntroToR}.


