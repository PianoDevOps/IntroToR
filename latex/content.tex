\section*{Overview}

This document contains general information for the "Hands-on Introduction to R" lectures.\newline

\section{Access to the \texttt{R} language interpretor and the \texttt{RStudio} IDE}

The lectures require the access to an \texttt{R} language interpretor and the \texttt{RStudio} IDE.
Below, you will find a few options:

\begin{enumerate}
\item If you have a valid CHPC account you can use CHPC's \href{http://ondemand.chpc.utah.edu/}{Ondemand Web Portal}.\newline
      After logging into OnDemand you can launch the \texttt{RStudioServer} application which contains \texttt{R} and \texttt{RStudio}.

\item You can apply for an account on \texttt{RStudioCloud} (which provides $25$ hours of free computing time per month).\newline
      \texttt{RStudioCloud} contains \texttt{R} and \texttt{RStudio}. 

\item You can also install \texttt{R} and \texttt{RStudio} on your machine.\newline
      The \texttt{R} command line interpretor can be downloaded from:
      \begin{itemize}
	      \item either \href{https://mran.microsoft.com/}{Microsoft R Open} (binaries for MS, Linux and Mac-OS).
  	 \item or the \href{https://cran.r-project.org/}{Comprehensive R Archive Network (CRAN)}.
      \end{itemize}			

      The \texttt{RStudio} Desktop IDE can be downloaded \href{https://www.rstudio.com/products/rstudio/download/}{here}.

\end{enumerate}

% Free R cheatsheets:
% https://www.rstudio.com/resources/cheatsheets/


\section{Obtaining the material}
All the course material can be obtained from: \href{https://github.com/wcardoen/IntroToR}{https://github.com/wcardoen/IntroToR}.

If you have \texttt{git} installed, you can obtain the material as follows:
\begin{verbatim}
git clone https://github.com/wcardoen/IntroToR.git
\end{verbatim}
Another option is to download the following \texttt{zip} file: \newline 
\href{https://github.com/wcardoen/IntroToR/archive/refs/heads/main.zip}{https://github.com/wcardoen/IntroToR/archive/refs/heads/main.zip}

